%%%%%%%%%%%%%%%%%
% This is an sample CV template created using altacv.cls
% (v1.6.5, 3 Nov 2022) written by LianTze Lim (liantze@gmail.com), based on the
% CV created by BusinessInsider at http://www.businessinsider.my/a-sample-resume-for-marissa-mayer-2016-7/?r=US&IR=T
%
%% It may be distributed and/or modified under the
%% conditions of the LaTeX Project Public License, either version 1.3
%% of this license or (at your option) any later version.
%% The latest version of this license is in
%%    http://www.latex-project.org/lppl.txt
%% and version 1.3 or later is part of all distributions of LaTeX
%% version 2003/12/01 or later.
%%%%%%%%%%%%%%%%

%% Use the "normalphoto" option if you want a normal photo instead of cropped to a circle
% \documentclass[10pt,a4paper,normalphoto]{altacv}

\documentclass[10pt,a4paper,ragged2e,withhyper]{altacv}

%% AltaCV uses the fontawesome5 package.
%% See http://texdoc.net/pkg/fontawesome5 for full list of symbols.

% Change the page layout if you need to
\geometry{left=1.25cm,right=1.25cm,top=1.5cm,bottom=1.5cm,columnsep=1.2cm}

% The paracol package lets you typeset columns of text in parallel
\usepackage{paracol}


% Change the font if you want to, depending on whether
% you're using pdflatex or xelatex/lualatex
\ifxetexorluatex
  % If using xelatex or lualatex:
  \setmainfont{Lato}
\else
  % If using pdflatex:
  \usepackage[default]{lato}
\fi

% Change the colours if you want to
\definecolor{VividPurple}{HTML}{3E0097}
\definecolor{SlateGrey}{HTML}{2E2E2E}
\definecolor{LightGrey}{HTML}{666666}
% \colorlet{name}{black}
% \colorlet{tagline}{PastelRed}
\colorlet{heading}{VividPurple}
\colorlet{headingrule}{VividPurple}
% \colorlet{subheading}{PastelRed}
\colorlet{accent}{VividPurple}
\colorlet{emphasis}{SlateGrey}
\colorlet{body}{LightGrey}

% Change some fonts, if necessary
% \renewcommand{\namefont}{\Huge\rmfamily\bfseries}
% \renewcommand{\personalinfofont}{\footnotesize}
% \renewcommand{\cvsectionfont}{\LARGE\rmfamily\bfseries}
% \renewcommand{\cvsubsectionfont}{\large\bfseries}

% Change the bullets for itemize and rating marker
% for \cvskill if you want to
\renewcommand{\itemmarker}{{\small\textbullet}}
\renewcommand{\ratingmarker}{\faCircle}

%% Use (and optionally edit if necessary) this .tex if you
%% want to use an author-year reference style like APA(6)
%% for your publication list
% \input{pubs-authoryear}

%% Use (and optionally edit if necessary) this .tex if you
%% want an originally numerical reference style like IEEE
%% for your publication list
\input{pubs-num}

%% publications.bib contains your publications
%\addbibresource{publications.bib}

\begin{document}
\name{Jonas Schäfer}
\tagline{Informatiker mit Interesse und Expertise für nachhaltige Entwicklung}
% Cropped to square from https://en.wikipedia.org/wiki/Marissa_Mayer#/media/File:Marissa_Mayer_May_2014_(cropped).jpg, CC-BY 2.0
%% You can add multiple photos on the left or right
\photoR{2.5cm}{jonas+earthly.png}
% \photoL{2cm}{Yacht_High,Suitcase_High}
\personalinfo{%
  % Not all of these are required!
  % You can add your own with \printinfo{symbol}{detail}
  \email{jonas.schaefer00@gmail.com}
  \phone{+49 176 69616761}
  % \mailaddress{}
  \location{Wadgassen, Deutschland}
  % \homepage{marissamayr.tumblr.com}
  % \twitter{@jonas4climate}
  \\
  \linkedin{jonas-schaefer}
  \github{jonas4climate} % I'm just making this up though.
%   \orcid{0000-0000-0000-0000} % Obviously making this up too.
  %% You can add your own arbitrary detail with
  %% \printinfo{symbol}{detail}[optional hyperlink prefix]
  % \printinfo{\faPaw}{Hey ho!}
  %% Or you can declare your own field with
  %% \NewInfoFiled{fieldname}{symbol}[optional hyperlink prefix] and use it:
  % \NewInfoField{gitlab}{\faGitlab}[https://gitlab.com/]
  % \gitlab{your_id}
	%%
  %% For services and platforms like Mastodon where there isn't a
  %% straightforward relation between the user ID/nickname and the hyperlink,
  %% you can use \printinfo directly e.g.
  % \printinfo{\faMastodon}{@username@instace}[https://instance.url/@username]
  %% But if you absolutely want to create new dedicated info fields for
  %% such platforms, then use \NewInfoField* with a star:
  % \NewInfoField*{mastodon}{\faMastodon}
  %% then you can use \mastodon, with TWO arguments where the 2nd argument is
  %% the full hyperlink.
  % \mastodon{@username@instance}{https://instance.url/@username}
}

\makecvheader

%% Depending on your tastes, you may want to make fonts of itemize environments slightly smaller
\AtBeginEnvironment{itemize}{\small}

%% Set the left/right column width ratio to 6:4.
\columnratio{0.6}

% Start a 2-column paracol. Both the left and right columns will automatically
% break across pages if things get too long.
\begin{paracol}{2}

\cvsection{Ehrenamt - Klima}

\cvevent{Leiter für Personal}{\href{https://climatescience.org}{ClimateScience}}{Jun 2022 -- Aktuell}{Remote}{\textbf{Vollzeit}}
\begin{itemize}
    \item \textbf{Aufbau der gesamten Personalabteilung inkl. Teams und  Strategie}
    \item Implementiert: Newsletter, Rekrutierungsprozess \& Outreach-Team, Schulungsprogramme und mehr für \textbf{500+ Freiwillige in 50 Ländern}.
    \item Beratung bei wichtigen Entscheidungen im Rahmen der Unternehmenskoordination
\end{itemize}

\divider

\cvevent{Leiter für Europäische Partnerschaften}{\href{https://climatescience.org}{ClimateScience}}{Dec 2021 -- Jun 2022}{Remote}{\textbf{Vollzeit}}
\begin{itemize}
    \item Initiierte \textbf{zwei Regierungskooperationen} in Europa
    \item Leitete ein Team von 10 Freiwilligen, das innerhalb von 3 Monaten \textbf{über 35 neue nationale Klimabildungsgemeinschaften gründete}.
\end{itemize}

\divider

\cvevent{Webentwickler \& Leiter für Deutsche Partnerschaften}{\href{https://climatescience.org}{ClimateScience}}{Dec 2019 --  Dec 2021}{Remote}{Teilzeit}
\begin{itemize}
    \item Programmierung der Frontend-Webseiten und des Authentifizierungsmoduls unserer \hrefUI{https://climatescience.org/de}{Website}, die \textbf{über 100.000 registrierte Benutzer} bedient.
    \item Akquirierte \textbf{mehr als 50.000 Dollar an Finanzmitteln} und führte die erste Regionalregierungspartnerschaft der Organisation in Europa ein.
\end{itemize}
 
\cvsection{Forschungserfahrung}

\cvevent{Industrie-Sommerpraktikum}{\href{https://royalsociety.org/}{Royal Society} \& \href{https://www.birmingham.ac.uk}{University of Birmingham}}{Jul 2020 -- Sep 2020}{Remote}{\textbf{Vollzeit}}
\begin{itemize}
    \item Betreut von \hrefU{https://www.jackiechappell.com/}{Dr. Jackie Chappell}, Senior Lecturer in Animal Behaviour \& Teamleiter der Cognitive Adaptations Research Group.
    \item Entwickelte eine \textbf{Deep Neural Network}-basiertes System zur Überwachung abnormaler Verhaltensweisen von in Gefangenschaft lebenden Tieren.
\end{itemize}

\cvsection{Bildung}

% \cvevent{M.Sc. in Computational Science}{University of Amsterdam \& VU (joint)}{Sep 2021 -- Dec 2021}{Amsterdam, NL}{Full-Time}

% \divider

\cvevent{B.Sc. (Hons) in Computer Science}{University of Birmingham}{Sep 2018 -- Jun 2021}{Birmingham, GB}{\textbf{Vollzeit}}
\begin{itemize}
    \item \textbf{Finale Note: \emph{First Class} - 4.0 GPA}
    \item \textbf{Schwerpunktbereiche}: Maschinelles Lernen, mathematische Modellierung, Maschinelles Sehen \& Robotik
    \item \textbf{Dissertationsprojekt "Expanding Standardisation in Optical Music Recognition":} Erweiterung einer aktuellen Standardisierung der optischer Musikerkennung durch eine neue  Notationsdarstellung, die die Verwendung jüngster \emph{Machine Learning}-Modelle ermöglicht
\end{itemize}

\newpage

% \cvsection{Publications}

% %% Specify your last name(s) and first name(s) as given in the .bib to automatically bold your own name in the publications list.
% %% One caveat: You need to write \bibnamedelima where there's a space in your name for this to work properly; or write \bibnamedelimi if you use initials in the .bib
% %% You can specify multiple names, especially if you have changed your name or if you need to highlight multiple authors.
% \mynames{Lim/Lian\bibnamedelima Tze,
%   Wong/Lian\bibnamedelima Tze,
%   Lim/Tracy,
%   Lim/L.\bibnamedelimi T.}
% %% MAKE SURE THERE IS NO SPACE AFTER THE FINAL NAME IN YOUR \mynames LIST

% \nocite{*}

% \printbibliography[heading=pubtype,title={\printinfo{\faBook}{Books}},type=book]

% \divider

% \printbibliography[heading=pubtype,title={\printinfo{\faFile*[regular]}{Journal Articles}}, type=article]

% \divider

% \printbibliography[heading=pubtype,title={\printinfo{\faUsers}{Conference Proceedings}},type=inproceedings]

%% Switch to the right column. This will now automatically move to the second
%% page if the content is too long.
\switchcolumn

% \cvsection{About me?}
% \begin{quote}
% ``Hello I'm Jonas:)''
% \end{quote}

\cvsection{Errungenschaft}

\cvhighlight{\faGlobeAfrica}{Aufbau einer führenden Klima-NGO}{Ich bin der zweitlängste Freiwillige bei \hrefUI{https://climatescience.org}{ClimateScience}, einer Wohltätigkeitsorganisation, die den Klimawandel leicht verständlich macht. Ich spielte eine \textbf{zentrale Rolle} bei der \textbf{Skalierung der Wohltätigkeitsorganisation} von Null auf jetzt \textbf{100.000+ Nutzer und 500+ Freiwillige}.}

\cvsection{Überzeugung}

\cvhighlight{\faHeartbeat}{Zwischenmenschliche Fähigkeiten}{sind der Schlüssel zu Selbstreflexion, effektiver Forschung und erfolgreicher Zusammenarbeit.}
% \divider

% \faChartLine \faFemale

\cvsection{Stärken}

\textbf{Allgemein}:\\
\vspace{3mm}
\begin{small}
    \cvtag{Führungsqualitäten}
    \cvtag{Projektleitung}
    \cvtag{Herzlichkeit}\\
    \cvtag{Zusammenarbeit}
    \cvtag{Enthusiasmus}
    \cvtag{Recherche}
\end{small}
\divider\smallskip

\textbf{Naturwissenschaften}:\\
\vspace{3mm}
\begin{small}
    \cvtag{Informatik}
    \cvtag{Künstliche Intelligenz}
    \cvtag{Mathematik}
    \cvtag{Kern- \& Astrophysik}
\end{small}

\cvsection{Sprachen}

\begin{itemize}
    \item Deutsch - Muttersprache
    \item Englisch - bilingual / C2
    \item Französisch - fortgeschritten / B1+
\end{itemize}

\cvsection{Konferenzen}
\begin{itemize}
    % \item Speaker: \hrefU{https://audiovisual.ec.europa.eu/en/video/I-235549?&lg=EN}{Youth Talk on Democracy and Climate} \hrefUI{https://audiovisual.ec.europa.eu/en/video/I-235549?&lg=EN}{Education} in \textbf{European Parliament} at \hrefU{https://www.levelup22.eu}{Level Up!} Conference, Brussels (Oct 2022)
    \item Nov 2022: Nominiert für die \textbf{Vertretung Deutschlands} bei \hrefU{https://coy17eg.com/}{COY17}, Ägypten
    \item Okt 2022: \hrefUI{https://audiovisual.ec.europa.eu/en/video/I-235549?&lg=EN}{Redner im \textbf{Europäischen Parlament}} zu Demokratie und Klima auf der \hrefU{https://www.levelup22.eu}{Jugendkonferenz} \hrefU{https://www.levelup22.eu}{Level Up!}, Brüssel
    \item Nov 2021: ClimateScience \textbf{Eventkoordinator} bei COP26, Glasgow. Veranstaltung \hrefUI{https://youtu.be/CjJ3cCRP9FU}{hier} ansehen
\end{itemize}

\cvsection{Zertifizierungen}
\begin{itemize}
    \item Dez 2022: Anfänger- und Fortgeschrittenen-Track des Herbstworkshops zu \textbf{Gesetzgebungsverfahren} bei \hrefU{https://www.europeanhorizons.org/}{European Horizons}
\end{itemize}

\newpage

% \cvsection{Referees}

% % \cvref{name}{email}{mailing address}
% \cvref{Prof.\ Alpha Beta}{Institute}{a.beta@university.edu}
% {Address Line 1\\Address line 2}

% \divider

% \cvref{Prof.\ Gamma Delta}{Institute}{g.delta@university.edu}
% {Address Line 1\\Address line 2}

\end{paracol}

\end{document}
