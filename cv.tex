%%%%%%%%%%%%%%%%%
% This is an sample CV template created using altacv.cls
% (v1.6.5, 3 Nov 2022) written by LianTze Lim (liantze@gmail.com), based on the
% CV created by BusinessInsider at http://www.businessinsider.my/a-sample-resume-for-marissa-mayer-2016-7/?r=US&IR=T
%
%% It may be distributed and/or modified under the
%% conditions of the LaTeX Project Public License, either version 1.3
%% of this license or (at your option) any later version.
%% The latest version of this license is in
%%    http://www.latex-project.org/lppl.txt
%% and version 1.3 or later is part of all distributions of LaTeX
%% version 2003/12/01 or later.
%%%%%%%%%%%%%%%%

%% Use the "normalphoto" option if you want a normal photo instead of cropped to a circle
% \documentclass[10pt,a4paper,normalphoto]{altacv}

\documentclass[11pt,a4paper,ragged2e,withhyper]{altacv}

%% AltaCV uses the fontawesome5 package.
%% See http://texdoc.net/pkg/fontawesome5 for full list of symbols.

% Change the page layout if you need to
\geometry{left=1.25cm,right=1.25cm,top=1.25cm,bottom=1.25cm,columnsep=1.2cm}

% The paracol package lets you typeset columns of text in parallel
\usepackage{paracol}


% Change the font if you want to, depending on whether
% you're using pdflatex or xelatex/lualatex
\ifxetexorluatex
  % If using xelatex or lualatex:
  \setmainfont{Lato}
\else
  % If using pdflatex:
  \usepackage[default]{lato}
\fi

% Change the colours if you want to
\definecolor{VividPurple}{HTML}{3E0097}
\definecolor{SlateGrey}{HTML}{2E2E2E}
\definecolor{LightGrey}{HTML}{666666}
% \colorlet{name}{black}
% \colorlet{tagline}{PastelRed}
\colorlet{heading}{VividPurple}
\colorlet{headingrule}{VividPurple}
% \colorlet{subheading}{PastelRed}
\colorlet{accent}{black}
\colorlet{logoAccent}{SlateGrey}
\colorlet{emphasis}{SlateGrey}
\colorlet{body}{LightGrey}

% Change some fonts, if necessary
% \renewcommand{\namefont}{\Huge\rmfamily\bfseries}
% \renewcommand{\personalinfofont}{\footnotesize}
% \renewcommand{\cvsectionfont}{\LARGE\rmfamily\bfseries}
% \renewcommand{\cvsubsectionfont}{\large\bfseries}

% Change the bullets for itemize and rating marker
% for \cvskill if you want to
\renewcommand{\itemmarker}{{\small\textbullet}}
\renewcommand{\ratingmarker}{\faCircle}

%% Use (and optionally edit if necessary) this .tex if you
%% want to use an author-year reference style like APA(6)
%% for your publication list
% % When using APA6 if you need more author names to be listed
% because you're e.g. the 12th author, add apamaxprtauth=12
\usepackage[backend=biber,style=apa6,sorting=ydnt]{biblatex}
\defbibheading{pubtype}{\cvsubsection{#1}}
\renewcommand{\bibsetup}{\vspace*{-\baselineskip}}
\AtEveryBibitem{%
  \makebox[\bibhang][l]{\itemmarker}%
  \iffieldundef{doi}{}{\clearfield{url}}%
}
\setlength{\bibitemsep}{0.25\baselineskip}
\setlength{\bibhang}{1.25em}


%% Use (and optionally edit if necessary) this .tex if you
%% want an originally numerical reference style like IEEE
%% for your publication list
\usepackage[backend=biber,style=ieee,sorting=ydnt]{biblatex}
%% For removing numbering entirely when using a numeric style
\setlength{\bibhang}{1.25em}
\DeclareFieldFormat{labelnumberwidth}{\makebox[\bibhang][l]{\itemmarker}}
\setlength{\biblabelsep}{0pt}
\defbibheading{pubtype}{\cvsubsection{#1}}
\renewcommand{\bibsetup}{\vspace*{-\baselineskip}}
\AtEveryBibitem{%
  \iffieldundef{doi}{}{\clearfield{url}}%
}


%% publications.bib contains your publications
%\addbibresource{publications.bib}

\begin{document}
\name{Jonas Schäfer}
\tagline{Science enthusiast \& generalist driven by sustainable impact}
% Cropped to square from https://en.wikipedia.org/wiki/Marissa_Mayer#/media/File:Marissa_Mayer_May_2014_(cropped).jpg, CC-BY 2.0
%% You can add multiple photos on the left or right
\photoR{2.5cm}{images/jonas.jpeg}
% \photoL{2cm}{Yacht_High,Suitcase_High}
\personalinfo{%
  % Not all of these are required!
  % You can add your own with \printinfo{symbol}{detail}
  \email{jonas.schaefer00@gmail.com}
  \phone{+31 6 1651 1643‬}
  % \mailaddress{}
  \location{Germany / Netherlands}
  % \homepage{marissamayr.tumblr.com}
  % \twitter{@jonas4climate}
  % \\
  \\
  \linkedin{jonas-schaefer}
  \github{jonas4climate} % I'm just making this up though.
%   \orcid{0000-0000-0000-0000} % Obviously making this up too.
  %% You can add your own arbitrary detail with
  %% \printinfo{symbol}{detail}[optional hyperlink prefix]
  % \printinfo{\faPaw}{Hey ho!}
  %% Or you can declare your own field with
  %% \NewInfoFiled{fieldname}{symbol}[optional hyperlink prefix] and use it:
  % \NewInfoField{gitlab}{\faGitlab}[https://gitlab.com/]
  % \gitlab{your_id}
	%%
  %% For services and platforms like Mastodon where there isn't a
  %% straightforward relation between the user ID/nickname and the hyperlink,
  %% you can use \printinfo directly e.g.
  % \printinfo{\faMastodon}{@username@instace}[https://instance.url/@username]
  %% But if you absolutely want to create new dedicated info fields for
  %% such platforms, then use \NewInfoField* with a star:
  % \NewInfoField*{mastodon}{\faMastodon}
  %% then you can use \mastodon, with TWO arguments where the 2nd argument is
  %% the full hyperlink.
  % \mastodon{@username@instance}{https://instance.url/@username}
}

\makecvheader

%% Depending on your tastes, you may want to make fonts of itemize environments slightly smaller
\AtBeginEnvironment{itemize}{\small}

%% Set the left/right column width ratio to 6:4.
\columnratio{0.5}

% Start a 2-column paracol. Both the left and right columns will automatically
% break across pages if things get too long.
\begin{paracol}{2}

\cvsectionsustainable{Sustainability Work}

\small{Since its inception in 2019, I have been instrumental in the growth of \hrefUI{https://climatescience.org}{\textbf{ClimateScience}} from a \textbf{small non-profit} to one of the \textbf{world's largest} climate change education organizations.}

\vspace{-0.5em}
\rule{\linewidth}{0.5pt}\par
\medskip\medskip

\cvevent{Sustainability Intern}{\href{https://prowin.net}{proWIN Winter GmbH}}{Mar -- Sep 2023}{}{Part-Time (20h/w)}
\begin{itemize}
    \item Consolidated the organization's sustainability report
    \item Advised on climate \& sustainability policy (\hrefU{https://www.efrag.org/lab6}{ESRS}).
\end{itemize}

% \rule{\linewidth}{1pt}\par
% \medskip

% \small{Since its inception in 2019, I have been instrumental in the growth of \hrefUI{https://climatescience.org}{\textbf{ClimateScience}} from a \textbf{small non-profit} to one of the \textbf{world's largest} climate change education organizations.}\\

\medskip\medskip

\cvevent{Head of Human Resources (Volunteer)}{\href{https://climatescience.org}{ClimateScience}}{Jun 2022 -- Mar 2023}{}{Full-Time}
\begin{itemize}
    \item Created HR department, set up HR team and strategy
    \item Initiated newsletters, recruiting pipelines, an outreach team, volunteer training programmes and assessment cycles for \textbf{500+ volunteers across 50 countries}.
    \item Member of Executive Coordination
\end{itemize}

\medskip

\cvevent{Web Developer \& EU Partnerships (Volunteer)}{\href{https://climatescience.org}{ClimateScience}}{Dec 2019 --  Jun 2022}{}{Part-Time (10h/w)}
% \begin{itemize}
%     \item Coded front-end web pages and the authentication module of our\\ \hrefUI{https://climatescience.org}{Web app} now serving over \textbf{100,000 registered users}.
%     \item Acquired about \textbf{\$100,000 in funding} for the organization and introduced the organisation's \textbf{first regional government partnership} in Europe.
%     \item Led a team of 10 volunteers to establish \textbf{over 35 new national climate education communities} within 3 months.
% \end{itemize}

\medskip

\cvsectionsustainable{Climate Conferences}
\begin{itemize}
    \item Nov 2022: Nominated to \textbf{represent German's youth} at \hrefU{https://coy17eg.com/}{COY17}, Egypt.
    \item Oct 2022: Speaker in \textbf{European Parliament} on Democracy and Climate at the \emph{Level Up!} conference, Brussels. View my speech \hrefUI{https://audiovisual.ec.europa.eu/en/video/I-235549?&lg=EN}{here}
    \item Jun 2022: Representation of ClimateScience at \hrefU{https://unfccc.int/SB56}{SB56}, Bonn
    \item Nov 2021: \textbf{Event coordination} for ClimateScience at \hrefU{https://unfccc.int/process-and-meetings/the-paris-agreement/the-glasgow-climate-pact-key-outcomes-from-cop26}{COP26}, Glasgow
\end{itemize}

\medskip

\cvsectionsustainable{Climate Certifications}
\begin{itemize}
    \item Dec 2022: Beginner \& Advanced Track \hrefU{https://www.europeanhorizons.org/}{European} \hrefU{https://www.europeanhorizons.org/}{Horizons} Fall \textbf{Policy Workshop} on climate policy
\end{itemize}

\medskip

\cvsectionlanguages{Languages}

\begin{enumerate}
    \item German (native)
    \item English (bilingual, C2)
    \item French (advanced, B1+)
\end{enumerate}

\newpage

% \cvsection{Publications}

% %% Specify your last name(s) and first name(s) as given in the .bib to automatically bold your own name in the publications list.
% %% One caveat: You need to write \bibnamedelima where there's a space in your name for this to work properly; or write \bibnamedelimi if you use initials in the .bib
% %% You can specify multiple names, especially if you have changed your name or if you need to highlight multiple authors.
% \mynames{Lim/Lian\bibnamedelima Tze,
%   Wong/Lian\bibnamedelima Tze,
%   Lim/Tracy,
%   Lim/L.\bibnamedelimi T.}
% %% MAKE SURE THERE IS NO SPACE AFTER THE FINAL NAME IN YOUR \mynames LIST

% \nocite{*}

% \printbibliography[heading=pubtype,title={\printinfo{\faBook}{Books}},type=book]

% \divider

% \printbibliography[heading=pubtype,title={\printinfo{\faFile*[regular]}{Journal Articles}}, type=article]

% \divider

% \printbibliography[heading=pubtype,title={\printinfo{\faUsers}{Conference Proceedings}},type=inproceedings]

%% Switch to the right column. This will now automatically move to the second
%% page if the content is too long.
\switchcolumn

% \cvsection{About me?}
% \begin{quote}
% ``Hello I'm Jonas:)''
% \end{quote}



% \divider

% \faChartLine \faFemale

% \cvsection{Strengths}

% \textbf{General}:\\
% \vspace{3mm}
% \begin{small}
%     \cvtag{Leadership}
%     \cvtag{Project Management}
%     \cvtag{Kindness}\\
%     \cvtag{Collaboration}
%     \cvtag{Optimism}
%     \cvtag{Research}
% \end{small}

% \medskip\medskip

% \textbf{Natural Sciences}:\\
% \vspace{3mm}
% \begin{small}
%     \cvtag{Computer Science}
%     \cvtag{Artificial Intelligence}
%     \cvtag{Mathematics}
%     \cvtag{Nuclear- \& Astrophysics}
% \end{small}

\cvsectioncomputational{Comp. Education}

\cvevent{M.Sc. in Computational Science (joint)}{University of Amsterdam \& Vrije University}{Sep 2023 -- Jul 2025}{}{Full-Time}
\begin{itemize}
    \item Grade: \textbf{7.5 GPA (currently)}
    \item \textbf{Focus areas}: scientific computing, complex  systems, computational biology, stochastic systems
\end{itemize}

\medskip

\cvevent{B.Sc. (Hons) in Computer Science}{University of Birmingham}{Sep 2018 -- Jun 2021}{}{Full-Time}
\begin{itemize}
    \item Grade: \textbf{First Class}
    \item \textbf{Focus areas}: Mathematical Modelling,  Machine Learning, Computer Vision \& Robotics
    \item \textbf{Thesis: "Expanding Standardisation in Optical Music Recognition"}
\end{itemize}

\cvsectioncomputational{Comp. Research}

\cvevent{Industry Summer Studentship}{\href{https://royalsociety.org/}{Royal Society} \& \href{https://www.birmingham.ac.uk}{University of Birmingham}}{Jul 2020 -- Sep 2020}{}{Full-Time}
\begin{itemize}
    \item Supervised by \hrefU{https://www.jackiechappell.com/}{Dr. Jackie Chappell}, Senior Lecturer in Animal Behaviour \& Team Leader of the Cognitive Adaptations Research Group.
    \item Designed a \textbf{Deep Neural Network}-based system to monitor abnormal behaviour of captive animals via CCTV-applied \textbf{pose estimation}.
\end{itemize}

\cvsectioncomputational{Comp. Projects}

\cvevent{Opinion formation as a complex system (Group)}{}{Jan - Feb 2024}{}{}
\begin{itemize}
    \item Complex system analysis of the opinion formation process in social structures, modelled using \hrefU{https://psycnet.apa.org/record/1982-01296-001}{social} \hrefU{https://psycnet.apa.org/record/1982-01296-001}{impact theory} on lattices and BA networks.
    \item Results include analysis of critical temperature, self-organized criticality, unification processes etc.
    \item Presentation and project code can be found   \hrefUI{https://github.com/jonas4climate/complex-systems-project}{here}.
\end{itemize}

\medskip

\cvevent{Modelling natural processes (Group)}{}{Jan 2024}{}{}
\begin{itemize}
    \item Numerical solutions to equations of \textbf{waves},\\ \textbf{diffusion}, \textbf{reaction-diffusion} and \textbf{Laplace}
    \item Modelling coral growth using \hrefU{https://www.sciencedirect.com/topics/mathematics/diffusion-limited-aggregation}{DLA}
    \item Find visualizations and project code \hrefUI{https://github.com/jonas4climate/modelling-natural-processes}{here}
\end{itemize}

\newpage

% \cvsection{Referees}

% % \cvref{name}{email}{mailing address}
% \cvref{Prof.\ Alpha Beta}{Institute}{a.beta@university.edu}
% {Address Line 1\\Address line 2}

% \divider

% \cvref{Prof.\ Gamma Delta}{Institute}{g.delta@university.edu}
% {Address Line 1\\Address line 2}

\end{paracol}

\end{document}
